\subsection{Лекция 4}

\begin{proof}
    Построим последовательность $X = S_0 \subseteq S_1 \subseteq \ldots$, где 

    \[ 
        S_{n+1} = S_n \cup \{\eta(e)| e \in E_n\},
    \]

    где $E_n$ и $\eta: E_n \rightarrow A$ определены следующим образом: 

    \[ 
        E_n = \{(\overline{a}, \varphi(\overline{x}, y))| \overline{a} \in S_n \text{ и } \mathbb{A} \models \exists y \varphi(\overline{a}, y)\}
    \]

    и $\mathbb{A} \models \varphi(\overline{a}, \eta(e))$ ($e \in E$). В качестве $B$ просто возьмём $\bigcup_n S_n$. Нужно проверить, что $|B| \leq |\text{For}_{\sigma}|$ -- это делается по индукции по $S_i$. $E_n$ по мощности не превосходит $\text{For}_{\sigma}$ посредством сравнения через $\text{For}_\sigma^2$, откуда и получаем требуемое. \ 
    
    Рассмторим теперь $\mathbb{B} = (B, I)$ с сигнатурой $\sigma$ и проверим, что $B$ замкнуто относительно интерпретаций элементов сигнатуры. Это получается несложно, а предикаты мы зададим как  

    \[ 
        P^{\mathbb{B}}(b_1, \ldots, b_n) \Lra P^{\mathbb{A}}(b_1, \ldots, b_n) = T.
    \]

    Осталось лишь проверить, что для любой формулы $\varphi(x_1, \ldots, x_k)$ и для любых значений переменных $(a_1, \ldots, a_k) = \overline{a} \in B$, тогда значение на этих элементах в $\mathbb{B}$ будет совпадать со занчением в $\mathbb{A}$: 

    \[ 
        \mathbb{B} \models \varphi(\overline{a}) \Longleftrightarrow \mathbb{A} \models \varphi(\overline{a}).
    \]

    Проверяется это, конечно, индукцией по построению формулы. Рассмотрим $\wedge, \neg$ и $\exists$, через них всё выражается и провреим для них. Конъюнкция -- очевидна, ровно как и отрицание. Интерес представляет существование. Пусть $\psi(\overline{x}) = \exists y \varphi(\overline{x}, y)$. Пусть для $\varphi$ уже доказано, что $\mathbb{B} \models \varphi(\overline{a}, c) \Longleftrightarrow \mathbb{A} \models(\overline{a}, c)$. Слева направо требуемое очевидно, а справа налево я проспал.
\end{proof} 

\begin{remark}
    На этом месте могло бы быть лирическое отступление про ZFC. 
\end{remark}

Пусть теперь $\mathbb{A} \equiv \mathbb{B}$, $\mathbb{A} \preceq \mathbb{B}$. $\tau$ называется \textit{обогащением} структуры $\sigma$, если последняя лежит в первой и дополнение непусто.

\begin{definition} \

    \begin{enumerate}
        \item Пусть $\mathbb{A}$ -- $\sigma$-структура. $\sigma_{\mathbb{A}} = \sigma \cup \{c_a | a \in A\}$, $c_a$ -- новые константные символы, причём $c_a \neq c_b$ при $a \neq b$. $D(\mathbb{A})$ -- множество атомарных формул сигнатуры $\sigma_{\mathbb{A}}$ и их отрицаний, истинных в $\mathbb{A}$ при интерпретации $\sigma_a \models a$. (\textit{диаграмма} $\mathbb{A}$) 
        \item \textit{Элементарная диаграмма} $\mathbb{A}$ -- это множество $D^*(\mathbb{A})$ всех предлжений $\sigma_{\mathbb{A}}$, истинных в $\mathbb{A}$. ($D(\mathbb{A}) \subseteq D^*(\mathbb{A})$)
    \end{enumerate}
\end{definition}

\begin{stat} \ 

    \begin{enumerate}
        \item Если $\mathbb{B} \models D(\mathbb{A})$, то $\mathbb{B}|_{\sigma}$ содержит подструктуту $\mathbb{A}' \subseteq \mathbb{B}|_{\sigma}$, такую что $\mathbb{A}' \simeq \mathbb{A}$. 
        \item Если $\mathbb{B} \models D^*(\mathbb{A})$, то $\mathbb{B}|_{\sigma}$ содержит элементарную подструктуру, изоморфную $\mathbb{A}$. 
    \end{enumerate}
\end{stat}

\begin{proof}
    *на доске рисуются картинки*
\end{proof} 

\begin{theorem}
    Пусть имеется бесконечная $\mathbb{A}$ -- $\sigma$-структура и $H \geq \max(|A|, |\text{For}_\sigma|)$. Тогда найдётся $\mathbb{B} \succeq \mathbb{A}$ можности хотя бы $H$.
\end{theorem}

\begin{proof}
    Рассмотрим $\sigma \mapsto \sigma_{\mathbb{A}} \models \tau = \sigma_{\mathbb{A}} \cup \{d_x | x \in H\}$ так, что $x \neq x' \Rightarrow d_x \neq d_{x'}$. И построим  

    \[ 
        \Gamma = D^*(A) \cup \{\neg(d_x = d_{x'}| x, x' \in H, x \neq x')\} 
    \]

    множество предложений сигнатуры $\tau$. Любое конечное $\Gamma_0 \subseteq \Gamma$ имеет модель, являющуюся $\tau$-расширением структуры $\mathbb{A}$ (легко проверяется). По теореме о компактности существует $\mathbb{C}$ -- $\tau$-структура, такая, что $\mathbb{C} \models \Gamma$. И как-то завершаем доказательство.
\end{proof}

\begin{definition}
    \textit{Теория} ($T$) -- множество всех предложений в структуре $\sigma$.
\end{definition}