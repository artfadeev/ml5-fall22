\documentclass[a4paper,11pt]{article}
\usepackage{../ml5}

\begin{document}

\begin{center}
	\Large Домашнее задание 3. Фильтрованные произведения \\ и компактность
\end{center}

\begin{enumerate}
	\item Говорят, что формула $\varphi(x_1,\ldots,x_k)$ условно фильтруется по фильтру $F$ на множестве $I$, если для любого семейства структур $\{\mathbb{A}_i\}_{i\in I}$ верно
	\[ \set*{i\ \middle\vert\ \mathbb{A}_i\models \varphi\lr*{a_1(i),\ldots,a_k(i)}}\in F\ \ \Lra\ \ 
	    \mathbb{A}_F\models\varphi\lr*{\segment*{a_1},\ldots,\segment*{a_k}}.\]

Говорят, что  $\varphi(x_1,\ldots,x_k)$  фильтруется по $F$, если выполнено
	\[ \set*{i\ \middle\vert\ \mathbb{A}_i\models \varphi\lr*{a_1(i),\ldots,a_k(i)}}\in F
	    \ \ \Longleftrightarrow\ \ 
	    \mathbb{A}_F\models\varphi\lr*{\segment*{a_1},\ldots,\segment*{a_k}}.\]

    \begin{enumerate}
        \item[(а)] Докажите, что если $\varphi,\psi$ условно фильтруются по $F$, то и $\varphi\land\psi$, $\forall x\,\varphi$, $\exists x\,\varphi$ также условно фильтруются по $F$. 

        \item[(б)] Докажите, что если $\varphi,\psi$  фильтруются по $F$, то и $\varphi\land\psi$, $\exists x\,\varphi$ также  фильтруются по $F$.
    \end{enumerate}


	\item Докажите, что любое фильтрованное произведение групп (предпорядков, частичных порядков) является группой (предпорядком, частичным порядком). 

Докажите, что для линейных порядков аналогичное утверждение, вообще говоря, неверно, но оно верно для ультрапроизведений.


	\item Теория (т. е. множество предложений) называется категоричной, если она имеет единственную модель (с точностью до изоморфизма). 

Докажите, что если теория категорична, то ее единственная модель конечна. 

Докажите, что любая конечная структура является единственной (с точностью до изоморфизма) моделью подходящей теории.


	\item Докажите, что если теория не имеет конечных моделей и категорична в некоторой мощности (т. е.  имеет единственную (с точностью до изоморфизма) модель этой мощности), то она полна (т. е. любое предложение или само следует из этой теории, или его отрицание следует из этой теории).


	\item Выведите из теоремы компактности, что существует полное упорядоченное поле (т.е. упорядоченное поле, в котором каждое непустое ограниченное сверху множество имеет супремум). Докажите, что полное упорядоченное поле единственно с точностью до изоморфизма.

\end{enumerate}
\end{document}


