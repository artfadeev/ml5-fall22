\documentclass[a4paper,11pt]{article}
\usepackage{../ml5}

\begin{document}

\begin{center}
	\Large Домашнее задание 5. Полнота и модельная полнота
\end{center}

\begin{enumerate}
\item Докажите, что для непротиворечивой теории равносильны следующие условия:
	\begin{enumerate}
		\item[–] теория полна,
		\item[–] любые две модели теории элементарно эквивалентны,
		\item[–] теория совпадает с теорией любой своей модели.
	\end{enumerate}

Докажите, что непротиворечивая теория $T$ модельно полна (т.~е. отношения $\subseteq$,~$\preceq$ совпадают на $\Models(T)$) тогда и только тогда, когда теория $T\cup D(\mathbb{A})$ полна для любой $\mathbb{A}\models T$.

Здесь $D(\mathbb{A})$~— множество истинных в $\mathbb{A}$ атомарных формул сигнатуры, пополненной константами для всех элементов $A$, и их отрицаний (см.~лекции).

\item \begin{enumerate}
	\item[(а)] Докажите, что теория плотного линейного порядка без наименьшего и наибольшего элемента является полной.
	\item[(б)] Полна ли теория плотного линейного порядка с наибольшим и наименьшим элементами?
	\item[(в)] Сколько пополнений есть у теории плотного линейного порядка? Перечислите их все.
\end{enumerate}

\item Являются  ли следующие теории полными (модельно полными)?
   \begin{enumerate}
	\item[(а)] Теория групп, теория абелевых групп
	\item[(б)] Теория абелевых групп без кручения
	\item[(в)] Теория полей, теория полей характеристики 0
	\item[(г)] Теория алгебраически замкнутых полей
	\item[(д)] Теория алгебраически замкнутых полей фиксированной характеристики
   \end{enumerate} 

\item \begin{enumerate}
	\item[(а)] Является ли полной (модельно полной) теория плотного неодноэлементного порядка  с наибольшим и наименьшим элементами?
	\item[(б)] Являются ли модельно полными теории $\Th(\mathbb{N};\leq)$ и $\Th(\mathbb{N};+)$?
\end{enumerate}

\item Докажите, что теория алгебраически замкнутых полей является модельно полной, но не полной. Перечислите все пополнения этой теории. (Предполагаются известными результаты из курса алгебры об алгебраически замкнутых полях.) 
\end{enumerate}

\end{document}


