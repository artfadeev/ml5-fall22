\documentclass[a4paper,11pt]{article}
\usepackage{../ml5}

\begin{document}

\begin{center}
	{\Large Домашнее задание 6. Выводимость.}

	{\it (26 октября\ \(\to\)\ 2 ноября)}
\end{center}


1. Докажите, что ``основные равносильности логики предикатов'' (т.е. равносильности, используемые при приведении формул к ДНФ и к предваренному виду) выводимы в двух версиях гильбертовского ИП и в генценовском исчислении секвенций.
\medskip

2. Докажите, что ``основные тавтологии'' (т.е. аксиомы гильбертовского ИП, не содержащие кванторов в явном виде), выводимы в генценовском исчислении секвенций. 
\medskip

3. Докажите, что из аксиом упорядоченных абелевых групп выводимы формулы, выражающие следующие утверждения:

если элемент $x$ положителен, то элементы $x+x$ и $x+x+x$ также положительны;

если элемент $x$ отрицателен, то элементы $x+x$ и $x+x+x$ также отрицательны;

квадрат любого ненулевого элемента подожителен;

порядок является плотным. 
\medskip

4. Докажите, что из аксиом арифметики Пеано выводимы формулы, выражающие следующие утверждения:

сложение ассоциативно и коммутативно;

умножение ассоциативно и коммутативно;

умножение дистрибутивно относительно сложения;

между $x$ и $x+1$ нет других элементов.
\medskip

5. Докажите, что из аксиом теории множеств ZFC выводимы формулы, выражающие следующие утверждения:

существует единственное пустое множество;

для любых двух множеств существуют и единственны их объединение, пересечение, и разность;

две упорядоченные пары (в смысле Куратовского) равны в точности тогда, когда равны их первые и вторые компоненты;

если  $x$ --- ординал, то  $x+1=x\cup\{x\}$ тоже ординал, и между ними нет других ординалов;

существует единственное множество натуральных чисел (определяемых как ординалы специального вида). 






\end{document}


