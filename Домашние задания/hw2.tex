\documentclass[a4paper,11pt]{article}
\usepackage{../ml5}

\begin{document}

\begin{center}
	\Large Домашнее задание 2. Определимость
\end{center}

\begin{enumerate}
	\item Постройте модели минимальной арифметики, в которых
	\begin{enumerate}
	   \item[(а)] Не выполняется коммутативность сложения,
	   \item[(б)] Бывает так, что \(x \not< x+1\).
	\end{enumerate}

	\item Докажите, что вещественное число определимо в структуре
		\[ (\mathbb{R};\ =,\ +,\ \cdot\, ,\ 0,\ 1)\]
	тогда и только тогда, когда оно алгебраическое. Охарактеризуйте вещественные числа, определимые в структуре
		\[ (\mathbb{R};\ =,\ +,\ 0,\ 1).\]

	\item Докажите, что комплексное число определимо в структуре
		\[ (\mathbb{C};\ =,\ +,\ \cdot\, ,\ 0,\ 1)\]
	тогда и только тогда, когда оно рациональное.

	\item Пусть $A$ --- $k$-буквенный алфавит, $k\geq2$. Определим бинарные отношения $\leq_p$, $\leq_s$, $\leq_i$, $\preceq$  на $A^*$ следующим образом:

	\begin{enumerate}
	   \item[–] $u\leq_pv$, если $ux=v$ для некоторого $x\in A^*$ (\(u\)~— {\it префикс} \(v\));
	   \item[–] $u\leq_sv$, если $xu=v$ для некоторого $x\in A^*$ (\(u\)~— {\it суффикс} \(v\));
	   \item[–] $u\leq_iv$, если $xuy=v$ для некоторых $x,y\in A^*$ (\(u\)~— {\it подслово} \(v\));
	   \item[–] $u\preceq v$, если $u$ получается из $v$ стиранием некоторых букв (\(u\)~— {\it подпоследовательность} \(v\)).
	\end{enumerate}

Докажите, что: 

\begin{enumerate}
	   \item[(а)] отношение $\leq_i$ определимо в $A^*$ через отношения  $\leq_p$ и $\leq_s$; 
	   \item[(б)]пустое слово определимо через любое из этих отношений; 
	   \item[(в)]множество всех слов фиксированной длины определимо через любое из этих отношений; 
	   \item[(г)]никакое фиксированное непустое слово не определимо через все эти отношения; 
	   \item[(д)]существует двухбуквенное слово, не определимое через все эти отношения и однобуквенные слова; 
	   \item[(е)] опишите  двухбуквенные слова, не определимые как в предыдущем вопросе.
\end{enumerate}

	\item Докажите, что любой элемент структуры $(A^*;\leq_i)$, обогащенной константами для всех слов длины не более двух, определим. Охарактеризуйте группу автоморфизмов структуры $(A^*;\leq_i)$. Докажите аналогичные результаты для отношения $\preceq$ вместо $\leq_i$.

\end{enumerate}

\end{document}
