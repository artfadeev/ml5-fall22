\documentclass[a4paper,11pt]{article}
\usepackage{../ml5}

\begin{document}

   \newcommand{\enumsep}{\vspace{-2.8mm}
   		\begin{enumerate}[itemsep=0.4mm,leftmargin=2.5mm]}

\begin{center}
	{\Large Домашнее задание 10–11. Рекурсивность и вычислимость.}

	{\it (16 ноября\ \(\to\)\ 23 ноября)}
\end{center}

\begin{enumerate}
	\item Докажите рекурсивность функций: \enumsep
		\item[(а)] \(\max\set*{x,y}\); максимум, минимум двух рекурсивных функций;
		\item[(б)] \( \left| x-y \right|\); \( \left\lfloor \sqrt{x} \right\rfloor\);
		\item[(в)] \[ f(x) = \begin{cases} n, & x=2n \\ 0, & x=2n+1; \end{cases} \]
		\item[(г)] \[ f\lr*{\bar{x}} = \begin{cases}
				g_1\lr*{\bar{x}}, & P_1\lr*{\bar{x}} = \text{И} \\
				g_2\lr*{\bar{x}}, & P_2\lr*{\bar{x}} = \text{И} \\
				\ldots & \ldots \\
				g_k\lr*{\bar{x}}, & P_k\lr*{\bar{x}} = \text{И}, \\
		\end{cases}\]
		здесь \(g_i\)~— рекурсивные функции, \(P_i\)~— дизъюнктные рекурсивные предикаты, \(\bigcup P_i = \bn^d\).
	\end{enumerate}

	\item Докажите, что любая рекурсивная функция \(f\) вычислима на машине Тьюринга. Докажите, что предикат $<$ рекурсивен.

	\item Докажите, что любая рекурсивная функция \(f\) представима как \(\l\)-выражение: существует терм \(F\) такой, что
		\[ f\lr*{n_1,\ldots,n_k} = n\ \ \Longleftrightarrow\ \ 
		   F\ \chN_1\ \ldots\ \chN_k \stackrel{\beta}{\longrightarrow} \chN.\]

	\item[4 а)] \stepcounter{enumi} \enumsep
		\item[ ] Докажите, что множество натуральных чисел разрешимо тогда и только тогда, когда оно само и его дополнение перечислимы. {\it (Теорема Поста)}
		\item[(б)] Докажите, что множество всех логических следствий перечислимой теории конечной сигнатуры перечислимо. \end{enumerate}

	\item Докажите, что следующие теории разрешимы: \enumsep
		\item[(а)] множество всех логических следствий аксиом нетривиальных делимых абелевых групп без кручения;
		\item[(б)] \ditto аксиом плотного линейного порядка; 
		\item[(в)] \ditto аксиом алгебраически замкнутых полей;
		\item[(г)] \ditto аксиом любой полной перечислимой теории. \end{enumerate}

	\item Пусть \(P \lr*{\bar{x}, y}\)~— рекурсивный предикат. Докажите, что рекурсивны также предикаты
	\[ \widetilde{P} \lr*{\bar{x}, z} \coloneqq \exists\,y<z\ P \lr*{\bar{x}, y},\qquad
	   \overset{\ \includegraphics[height=1ex]{wave}}{P} \lr*{\bar{x}, z}
	      \coloneqq \forall\,y<z\ P \lr*{\bar{x}, y}.\]

\end{enumerate}

\end{document}
