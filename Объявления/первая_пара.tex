\documentclass[a4paper,12pt]{article}
\usepackage[margin=2.5cm,left=2.35cm,top=1.4cm]{geometry}
\usepackage{amsmath,amssymb,mathtools,xcolor}
\usepackage{enumerate,enumitem,indentfirst}
\usepackage{xspace,sectsty,hyperref}
\usepackage{makecell}
\usepackage[russian]{babel}

\usepackage{mathspec}

\setmainfont[
	Path = f/,
	BoldFont=lb.ttf,
	ItalicFont=li.ttf,
	BoldItalicFont=lbi.ttf
		]{l.ttf}
\setsansfont[
	Path = f/,
	BoldFont=pb.ttf,
	ItalicFont=pi.ttf,
	BoldItalicFont=pbi.ttf
		]{p.ttf}
		
\setmathfont(Digits)[Path = f/]{csr.ttf}
\setmathfont(Latin)[Path = f/]{csi.ttf}
\setmathfont(Greek)[Path = f/, Uppercase]{t.ttf}
\setmathfont(Greek)[Path = f/, Lowercase]{ti.ttf}

\setmonofont[Path = f/]{pmono.ttf}
 \newcommand{\bz}{\mathbb Z} \newcommand{\br}{\mathbb R}
\newcommand{\vfi}{φ}
\newcommand{\divsby}{\mathop{\rlap{.}\rlap{\raisebox{0.55ex}{.}}\raisebox{1.1ex}{.}}}
\renewcommand{\ll}{\left(} \newcommand{\rr}{\right)}
\newcommand{\lag}{\left\langle} \newcommand{\rag}{\right\rangle}
\newcommand{\legendre}[2]{\ensuremath{\left( \frac{#1}{#2} \right)}}
\allsectionsfont{\normalfont\sffamily\bfseries}
\parindent=0.5in \parskip=0.1in \linespread{1.11}

\setlist{leftmargin=0.35in}
\hypersetup{nolinks=true}

\begin{document}

\begin{center} {\LARGE\bf \ \\
	Математическая логика, 3 курс, \\
	Осенний семестр 2022 \\
	Экзамен, практики, электронный конспект \\}
\end{center}

\section{Разделы курса}
\begin{enumerate}
	\item[–] Логика предикатов
	\item[–] Неразрешимость и неполнота
	\item[–] Введение в вычислимость
\end{enumerate}

\section{Экзамен}

Экзамен состоит из:

\begin{enumerate}
	\item[–] Один вопрос по теории с подробным ответом и доказательством (из {\it списка нетривиальных фактов);}
	\item[–] Четыре случайных вопроса по теории, только формулировки, без подготовки (из {\it списка вообще всех определений и утверждений);}
	\item[–] Задача, аналогичная задачам с практики. Топ-10 лучших студентов по итогам практических занятий (по всем группам вместе) освобождаются от задачи.
\end{enumerate}

\section{Практики}

	Каждую пару практики на дом выдаются задачи для самостоятельной работы. На следующей паре собирается информация: кто из студентов решил какие задачи. По каждой задаче кто-то из заявившихся студентов вызывается к доске.

	Студент, заявивший, что решил задачу дома, получает за это \(1\) балл. Если студент был вызван к доске и рассказал верное решение, \(1\) заменяется на \(2\). Если студент был вызван к доске и рассказал ошибочное решение, \(1\) заменяется на \(0\)–\(0.5\) (в зависимости от размера дыры). Если выяснилось, что студент соврал, заявив о наличии у себя решения, \(1\) заменяется на \(-50\).

	Чтобы {\it закрыть раздел,} необходимо набрать по нему не менее \(0.6 n\) баллов, где \(n\)~— количество выданных по нему (в сумме за все пары) задач для самостоятельной работы. Если не закрыть хотя бы один раздел (из трёх), оценку \(A\) за экзамен получить будет нельзя.

\section{Электронный конспект}

	Данный курс читается данным лектором в СПбГУ впервые, и материал курса не покрывается литературой полностью. Нам хотелось бы сохранить какого-то вида пособие по окончании данного курса, поэтому мы настоятельно просим вызваться добровольца писать электронный конспект.

	Хороший, подробный электронный конспект~— это ваша инвестиция в ус-\linebreak пешную сдачу экзамена. Надеемся, вас мотивирует его писать и то, что Б.~А. обязуется регулярно в него смотреть и вносить необходимые правки / что-то дописывать.\hspace{4cm} \textcolor{white}{\cite{shen2,shen3,katlend,lavrmaks,shenfield}}

\section{Github}

	Репозиторий на Github, где будут храниться файлы с домашними заданиями (хотя на бумажке мы их тоже будем раздавать), конспект и прочие материалы:

  \begin{center}
	\url{https://github.com/boris-a-zolotov/ml5-fall22}
  \end{center}

\bibliographystyle{plain}
\bibliography{../ml}


\end{document}
