\documentclass[a4paper,11pt]{article}
\usepackage{../ml5}

\setlist{itemsep=0.4mm,leftmargin=*}

\title{Системы аксиом, тавталогии, правила вывода}
\date{\today}
\author{В.~Л.~Селиванов}

\begin{document} \maketitle \tableofcontents

\newpage
\section{Основные равносильности}

\begin{enumerate}
\item $(\varphi\rightarrow\psi)\equiv(\neg\varphi\lor\psi)$; 

\item $\neg\neg\varphi\equiv\varphi$;

\item $\neg (\varphi \land \psi)\equiv (\neg\varphi\lor \neg\psi)$;

\item $\neg(\varphi \lor \psi) \equiv(\neg\varphi \land \neg\psi)$;

\item $(\varphi\land\psi) \equiv(\psi \land \varphi)$; 

\item $(\varphi \lor \psi)\equiv(\psi \lor \varphi)$;

\item $\varphi\land(\psi\land\theta)\equiv(\varphi\land\psi)\land\theta$;

\item $\varphi\lor (\psi\lor \theta) \equiv(\varphi \lor \psi) \lor \theta$;

\item $\varphi\land(\psi\lor\theta)\equiv(\varphi\land\psi)\lor(\varphi\land\theta)$; 

\item $\varphi\lor(\psi\land\theta)\equiv(\varphi\lor\psi)\land(\varphi\lor\theta)$.\medskip

\item $\neg(\forall x \varphi)\equiv \exists x (\neg\varphi)$; 

\item $\neg (\exists x \varphi)\equiv \forall x (\neg\varphi)$;

\item $\psi \land \forall x \varphi\equiv \forall x (\psi\land\varphi)$; 

\item $\psi\lor \exists x \varphi \equiv \exists x (\psi \lor \varphi)$;

\item $\psi \lor \forall x \varphi\equiv \forall x (\psi\lor \varphi)$; 

\item $\psi \land\exists x \varphi \equiv\exists x (\psi \land \varphi)$
($x$ не входит свободно в $\psi$);

\item $\forall x \varphi(x) \equiv\forall y \varphi(y)$; 

\item $\exists
x\varphi(x) \equiv \exists y\varphi(y)$
($y$ не входит в $\varphi$).
\end{enumerate}


\newpage
\section{Основные тавтологии}

\begin{enumerate}
\item $\varphi\rightarrow(\psi\rightarrow\varphi)$; 

\item $(\varphi\rightarrow\psi) \rightarrow \bigl((\varphi\rightarrow (\psi
\rightarrow \theta))\rightarrow
(\varphi \rightarrow\theta)\bigr)$;
\medskip

\item $\varphi \rightarrow (\psi\rightarrow (\varphi \land \psi))$;

\item $(\varphi
\land \psi)\rightarrow \varphi$;

\item $(\varphi \land \psi) \rightarrow\psi$; 
\medskip

\item $\varphi
\rightarrow
(\varphi \lor \psi)$;

\item $\psi \rightarrow (\varphi \lor \psi)$;

\item $(\varphi
\rightarrow \theta) \rightarrow \Bigl((\psi \rightarrow \theta)
\rightarrow \bigl((\varphi \lor \psi) \rightarrow
\theta\bigr)\Bigr)$;
\medskip

\item $(\varphi \rightarrow \psi) \rightarrow \bigl((\varphi \rightarrow
\neg\psi) \rightarrow \neg\varphi\bigr)$; 

\item $\neg\neg\varphi\rightarrow \varphi$.
\end{enumerate}


\section{Гильбертовское исчисление предикатов}

\subsection{Аксиомы}

\begin{enumerate}
	\item Тавтологии сигнатуры $\sigma$;

	\item Кванторные аксиомы:
\[ \forall x \varphi(x)\rightarrow \varphi(t) \text{\ \ и\ \ } \varphi(t)\rightarrow \exists x \varphi(x). \]
\end{enumerate}

\subsection{Аксиомы равенства}

\begin{enumerate}
\item $\forall x (x = x)$, 

\item $\forall x \forall y(x = y
\rightarrow y = x)$, 

\item $\forall x \forall y\forall z (x = y \land y
= z \rightarrow x = z)$, 

\item $\forall x_1 \forall y_1\ldots \forall
x_n \forall y_n (x_1 = y_1 \land \ldots \land x_n = y_n
\rightarrow f(x_1,\ldots,x_n) = f(y_1,\ldots, y_n))$, 

\item $\forall x_1
\forall y_1\ldots\forall x_n \forall y_n (x_1 = y_1 \land \ldots
\land x_n = y_n \land P(x_1,\ldots, x_n) \rightarrow
P(y_1,\ldots, y_n))$.
\end{enumerate}

\subsection{Правила вывода}

\[ \dfrac{\varphi,\ \ \varphi\rightarrow\psi}{\psi},\quad
\dfrac{\psi\rightarrow \varphi(y)}{\psi\rightarrow\forall
x\varphi(x)},\quad
\dfrac{\varphi(y)\rightarrow\psi}{\exists
x\varphi(x)\rightarrow\psi},\]
где $y$~— переменная, не входящая свободно в нижнюю формулу.

\noindent Вариант $\text{ИП}^*_\sigma$ получается, если вместо всех тавтологий берутся только {\it основные} тавтологии.

\pagebreak

\section{Генценовское исчисление предикатов (без равенства)}

\subsection{Аксиомы}

\[\Gamma,\varphi\vdash \Delta,\varphi\]

\subsection{Правила вывода}

\begin{align*}
&\frac{\Gamma,\varphi,\psi\vdash\Delta}{\Gamma,\varphi\land\psi\vdash\Delta},&&
\frac{\Gamma\vdash\Delta,\varphi;\;\Gamma\vdash\Delta,\psi}{\Gamma\vdash\Delta,\varphi\land\psi},
\\[0.3cm]
&\frac{\Gamma,\varphi\vdash\Delta;\;\Gamma,\psi\vdash\Delta}{\Gamma,\varphi\lor\psi\vdash\Delta},&&
\frac{\Gamma\vdash\Delta,\varphi,\psi}{\Gamma\vdash\Delta,\varphi\lor\psi},
\\[0.3cm]
&\frac{\Gamma\vdash\Delta,\varphi;\;\Gamma,\psi\vdash\Delta}{\Gamma,\varphi\rightarrow\psi\vdash\Delta},&&
\frac{\Gamma,\varphi\vdash\Delta,\psi}{\Gamma\vdash\Delta,\varphi\rightarrow\psi},
\\[0.3cm]
&\frac{\Gamma\vdash\Delta,\varphi}{\Gamma,\neg\varphi\vdash\Delta},&&
\frac{\Gamma,\varphi\vdash\Delta}{\Gamma\vdash\Delta,\neg\varphi},
\\[0.3cm]
&\frac{\Gamma,\varphi(t)\vdash\Delta}{\Gamma,\forall x\varphi(x)\vdash\Delta}, &&
\frac{\Gamma\vdash\Delta,\varphi(y)}{\Gamma\vdash\Delta,\forall x\varphi(x)},
\\[0.3cm]
&\frac{\Gamma,\varphi(y)\vdash\Delta}{\Gamma,\exists x\varphi(x)\vdash\Delta}, &&
\frac{\Gamma\vdash\Delta,\varphi(t)}{\Gamma\vdash\Delta,\exists x\varphi(x)},
\\[0.6cm]
&\frac{\Gamma\vdash\Delta,\varphi;\;\Gamma,\varphi\vdash\Delta}{\Gamma\vdash\Delta}
\end{align*}

\newpage
\section{Минимальная арифметика и арифметика Пеано}

\begin{enumerate}
	\item $0 + 1 = 1$;

	\item $\forall x \neg(x + 1 = 0)$;

	\item $\forall x \forall y (x + 1 = y +1 \rightarrow x = y)$; 

	\item $\forall x (x+0 = x)$;

	\item $\forall x \forall y (x + (y + 1)= (x + y) + 1)$; 

	\item $\forall x (x\cdot 0 = 0)$;

	\item $\forall x \forall y (x\cdot (y + 1)= (x \cdot y) + x)$;  

	\item $\forall x \neg (x <0)$;

	\item $\forall x \forall y (x < y \vee x = y \vee y < x)$;  

	\item  $\forall x\forall y (x < y + 1 \leftrightarrow (x < y \vee x = y))$.
\end{enumerate}

\noindent Арифметика Пеано ПА получается из МА добавлением схемы аксиом индукции:
\[(\varphi(0) \wedge \forall
x(\varphi(x) \rightarrow \varphi(x + 1))) \rightarrow \forall x
\varphi(x),\]
где $\varphi(x)$ --- любая формула сигнатуры MA.
 
\section{Аксиомы теории множеств}

\begin{enumerate}
\item $\exists x(x=x)$.

\item $\forall u(u\in X\leftrightarrow u\in Y)\to X=Y$ (аксиома объёмности).

\item $\forall u\forall v\exists X\forall z(z\in X\leftrightarrow z=u\lor z=v)$ (существование мн-ва из двух элементов).

\item $\forall X\exists Y\forall u(u\in Y\leftrightarrow u\in X\land\varphi(u))$ (аксиома выделения).

\item $\forall X\exists Y\forall u\forall z(u\in z\land z\in X\to u\in Y)$ (существование объединения семейства мн-в).

\item $\forall X\exists Y\forall u(u\in Y\leftrightarrow u\subseteq X)$ (существование множества подмножеств).

\item $\forall x\forall y\forall y'(\varphi(x,y)\land\varphi(x,y')\to y=y')$

$\to\forall X\exists Y\forall x\forall y(x\in X\land\varphi(x,y)\to y\in Y)$ (существование образа функции).

\item $\exists Y(\emptyset\in Y\land\forall y(y\in Y\to y\cup\{y\}\in Y))$ (существование бесконечного мн-ва).

\item $\forall X(X\not=\emptyset\to\exists x(x\in X\land\forall u(u\in x\to u\not\in X)))$ (аксиома регулярности).

\item $\forall X\exists f((f:(P(X)\setminus\{\emptyset\})\to X)\land\forall Y(Y\subseteq X\land Y\not=\emptyset\to f(Y)\in Y))$.
\end{enumerate}




\end{document}
